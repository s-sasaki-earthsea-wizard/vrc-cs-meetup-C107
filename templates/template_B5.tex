%%%%%%%%%%%
% uplatex %
%%%%%%%%%%%

% ==== jlreq + jlreq-trimmarks で B5 / 3 mm 塗り足し付き ====
\documentclass[b5paper,twocolumn]{jlreq}      % 和文組版用クラス(B5版・二段組)

% trimmarks:トンボを描く
% platex,dvipdfmx:Cloud LaTeX の uplatex+dvipdfmx に合わせる
% bleed_margin:塗り足し幅(四辺とも 3 mm)
% show={trimmarks,digital}:PDF に TrimBox/BleedBox を自動設定
\ifdefined\notrimmarks
  % トリムマークなし版:jlreq-trimmarksを読み込まない
\else
  % トリムマークあり版:jlreq-trimmarksを読み込む
  \usepackage[
    platex,dvipdfmx,
    bleed_margin={top=3mm,bottom=3mm,fore-edge=3mm,gutter=3mm},
    show   ={trimmarks,digital}
  ]{jlreq-trimmarks}
\fi




% 2025-07-03 amsmathを追加
\usepackage{amsmath}

% ──────────── ここから OCG の直打ち ────────────
\usepackage{xcolor}
% ドキュメント開始時に「Memo」という OCG(Optional Content Group)を作成、
% print 時に必ず非表示にする設定
\AtBeginDocument{%
  \special{pdf:optionalcontent create Memo << /PrintState /OFF >>}%
}

% 画面のみ表示のメモ用コマンド
\newcommand{\printnote}[2][cyan!50]{%
  \special{pdf:optionalcontent begin Memo}% Memo レイヤ開始
    \textcolor{#1}{#2}% 好きな色でメモ
  \special{pdf:optionalcontent end Memo}% Memo レイヤ終了
}
% ────────────────────────────────────




\usepackage{xurl}
\usepackage[hidelinks]{hyperref}
\usepackage{ascmac}

% fancyhdr でヘッダー/フッターをカスタマイズ
\usepackage{fancyhdr}
\fancypagestyle{myhead}{%
  \fancyhf{}                                      
  \fancyhead[L]{\footnotesize タイトル} 
  \fancyhead[R]{\footnotesize \rightmark}         
  \fancyfoot[C]{\thepage}                         
  \renewcommand{\headrulewidth}{0.4pt}            
}
\pagestyle{myhead}

\makeatletter
% 番号なしで右ヘッダにセクション名だけを流す
\renewcommand{\sectionmark}[1]{%
  \markright{\thesection\hspace{1em}#1}
}
\makeatother

% ── プリアンブル中,\usepackage{fancyhdr} や \makeatletter 周りのあとあたりに追加 ──
\usepackage{titling}
% デフォルトだと \droptitle=0pt なので,下にずれているぶんだけマイナスにします。
% たとえば「標準位置から上に10mmずらす」なら:
\setlength{\droptitle}{-10mm}
% ────────────────────────────────────────────

\usepackage{tabto}

% 星付きセクションでも目次・ヘッダに登録するマクロ
\usepackage{etoolbox}
\newcommand{\sectstar}[1]{%
  \phantomsection%
  \section*{#1}%
  \addcontentsline{toc}{section}{#1}%
  \markright{#1}%
}

\newcommand{\columnauthor}[1]{%
  \begin{flushright}
    {\small 執筆:#1}%
  \end{flushright}
}

\usepackage{float}
\usepackage[export]{adjustbox}  % プリアンブルに追加(まだなら)
\jlreqsetup{caption_align={center}} % {center,*left} とすると長い場合のみ左寄せ

% 枠付き画像を一発で呼び出すマクロ
\newcommand{\ImgFrame}[2][]{
  \fbox{\includegraphics[width=#1,keepaspectratio]{#2}}
}
% 枠線と余白を好みで調整
\setlength{\fboxrule}{0.1pt} % 線の太さ
\setlength{\fboxsep}{0pt}    % 画像–枠の間

% マクロにしておくと楽
\newcommand{\framedimg}[2][]{%
  \fbox{\includegraphics[#1]{#2}}%
}

\usepackage{xurl}





\title{{\large タイトル} \\ \vspace{2mm} {サブタイトル}}
\author{著者名 著}
\date{YYYY年MM月DD日} % 発行日






\begin{document}
% ----裏表紙--------------------
\clearpage              % 直前のページを終わらせる
% \printnote{(裏表紙)}
\thispagestyle{empty}   % ヘッダ・フッタ(ページ番号)を消す
\null                  % なにもない本文(高さ 0 の箱)


\newpage               % 次ページへ
\pagenumbering{arabic}

\newpage


\maketitle

% 目つぎに「タイトル」を入れるかどうか迷う
% \phantomsection
% \addcontentsline{toc}{section}{タイトル}

\vspace{20mm}


\sectstar{タイトル}

ああああああああああああああああ \\
いいいいいいいいいいいいいいいいい \\
ううううううううううううううううう \\
えええええええええええええええええ \\
おおおおおおおおおおおおおおおおお \\
かかかかかかかかかかかかかかかかか \\
ききききききききききききききききき \\
くくくくくくくくくくくくくくくくく \\
けけけけけけけけけけけけけけけけけ \\
こここここここここここここここここ \\
さささささささささささささささささ \\
ししししししししししししししししし \\
すすすすすすすすすすすすすすすすす \\
せせせせせせせせせせせせせせせせせ \\
そそそそそそそそそそそそそそそそそ \\
たたたたたたたたたたたたたたたたた \\
ちちちちちちちちちちちちちちちちち \\
つつつつつつつつつつつつつつつつつ \\
ててててててててててててててててて \\
ととととととととととととととととと \\
ななななななななななななななななな \\
ににににににににににににににににに \\
ぬぬぬぬぬぬぬぬぬぬぬぬぬぬぬぬぬ \\
ねねねねねねねねねねねねねねねねね \\
ののののののののののののののののの \\
ははははははははははははははははは \\
ひひひひひひひひひひひひひひひひひ \\
ふふふふふふふふふふふふふふふふふ \\
へへへへへへへへへへへへへへへへへ \\
ほほほほほほほほほほほほほほほほほ \\
ままままままままままままままままま \\
みみみみみみみみみみみみみみみみみ \\
むむむむむむむむむむむむむむむむむ \\
めめめめめめめめめめめめめめめめめ \\
ももももももももももももももももも \\
ややややややややややややややややや \\
ゆゆゆゆゆゆゆゆゆゆゆゆゆゆゆゆゆ \\
よよよよよよよよよよよよよよよよよ \\
わわわわわわわわわわわわわわわわわ \\
ををををををををををををををををを \\
んんんんんんんんんんんんんんんんん \\
あああああああああああああああああ \\
いいいいいいいいいいいいいいいいい \\
ううううううううううううううううう 

\columnauthor{著者名(x\@for\_your\_sns\_account)}

\clearpage


% 冬コミ2024にてCS集会から初めての本が出版された。2冊発行されており、それぞれ「夜鍋ヨナ計算機科学」、「独奏 CS集会LT論評 1」というタイトルである。
% \\
% 「夜鍋ヨナ計算機科学」では、夜鍋ヨナ氏がCS集会にて4bitコンピューター設計をテーマにしたLTのスライド資料を編集したものである。当初1000Pを超えてしまいそうになり、急遽ページ数を削減して100Pに収め、10dBの低減を見せた。

% 夏コミ2025もありがたく当選させていただいたので、新刊を出すことにした。今回は2名体制での執筆となる。誘ったら必ずOKを出すよいフレンドが居て助かった。このまま冬コミにも書いてもらおう。
% さて、夏コミ2025の新刊は二人で執筆したため、前半と後半で異なる味わいになるはずである。そのあたりは読者も楽しみにしていただきたい。

% \newpage

\phantomsection
\addcontentsline{toc}{section}{目次}


% 目次
\tableofcontents
\clearpage


\clearpage

\sectstar{まえがき 著者名}
ここに前書きを書く

\columnauthor{著者名(x\@for\_your\_sns\_account)}
\clearpage


\section{第N章の名前}

\begin{screen}
    あればここにキャッチコピーを書く
\end{screen}
\newline

ああああああああああああああああ \\
いいいいいいいいいいいいいいいいい \\
ううううううううううううううううう \\
えええええええええええええええええ \\
おおおおおおおおおおおおおおおおお \\
かかかかかかかかかかかかかかかかか \\
ききききききききききききききききき \\
くくくくくくくくくくくくくくくくく \\
けけけけけけけけけけけけけけけけけ \\
こここここここここここここここここ \\
さささささささささささささささささ \\
ししししししししししししししししし \\
すすすすすすすすすすすすすすすすす \\
せせせせせせせせせせせせせせせせせ \\
そそそそそそそそそそそそそそそそそ \\
たたたたたたたたたたたたたたたたた \\
ちちちちちちちちちちちちちちちちち \\
つつつつつつつつつつつつつつつつつ \\
ててててててててててててててててて \\
ととととととととととととととととと \\
ななななななななななななななななな \\
ににににににににににににににににに \\
ぬぬぬぬぬぬぬぬぬぬぬぬぬぬぬぬぬ \\
ねねねねねねねねねねねねねねねねね \\
ののののののののののののののののの \\
ははははははははははははははははは \\
ひひひひひひひひひひひひひひひひひ \\
ふふふふふふふふふふふふふふふふふ \\
へへへへへへへへへへへへへへへへへ \\
ほほほほほほほほほほほほほほほほほ \\
ままままままままままままままままま \\
みみみみみみみみみみみみみみみみみ \\
むむむむむむむむむむむむむむむむむ \\
めめめめめめめめめめめめめめめめめ \\
ももももももももももももももももも \\
ややややややややややややややややや \\
ゆゆゆゆゆゆゆゆゆゆゆゆゆゆゆゆゆ \\
よよよよよよよよよよよよよよよよよ \\
わわわわわわわわわわわわわわわわわ \\
ををををををををををををををををを \\
んんんんんんんんんんんんんんんんん \\
あああああああああああああああああ \\
いいいいいいいいいいいいいいいいい \\
ううううううううううううううううう \\
えええええええええええええええええ \\
おおおおおおおおおおおおおおおおお \\
かかかかかかかかかかかかかかかかか \\
ききききききききききききききききき \\
くくくくくくくくくくくくくくくくく \\
けけけけけけけけけけけけけけけけけ \\
こここここここここここここここここ \\
さささささささささささささささささ \\
54行*17文字=918文字/page(除コピー)
\columnauthor{夜鍋ヨナ(twitter@yonabeyona)}

\clearpage




\sectstar{あとがき 著者名}
\markright{あとがき 著者名}

ここにあとがきを書く



\columnauthor{夜鍋ヨナ(twitter@yonabeyona)}
\clearpage


\onecolumn  % 一段組に戻す
\subsection*{CS集会について}
\markright{CS集会について}

CS集会では、コンピューターサイエンスを中心とした、VRChat上の技術学術交流コミュニティです。
VRChat上で隔週火曜日に開催され、コンピューターにまつわる様々な分野の研究者・エンジニア・学生が集まり、知識の共有と交流を行っています。

\subsection*{活動内容}
CS集会では毎回のLTセッションを設けています。毎週誰かが登壇して面白いテーマの発表が聞けます。
発表の後は質疑応答や、より深堀りした内容の雑談・討論をしている様子が散見されます。

\subsection*{参加について}
参加は基本的に自由です。VRChatアカウントがあれば、どなたでもご参加いただけます。
開催情報は技術学術イベントHubで確認できます。 \url{https://vrc-ta-hub.com/community/11/}

\subsection*{過去の活動実績 $\cdot$ 今後の展望}
これまでに数多くのLTが行われ、コンピューターサイエンスはもちろん、光学、物理学、コンパイラ、果てにはゲームボーイなど
幅広い分野の発表が行われてきました。
\\
CS集会は「すべての学問はコンピューターサイエンスとつながりがある」という理念のもと、
分野を超えた知識交流の場として発展を続けます。
新たな参加者、発表者を常に歓迎しています。一緒に学術コミュニティを盛り上げていきましょう。
\\
興味をお持ちの方は、ぜひご参加下さい。


\newpage

\thispagestyle{empty}


% \begin{figure}[htbp]
%   \centering
%   % \linewidth ≒ \textwidth(単カラムの場合)。どちらでも OK
%   \makebox[\textwidth][c]{%
%     \includegraphics[width=1.2\textwidth]{image/Hub.eps}}
%   \caption{技術・学術系イベント Hub}

%   % \linewidth ≒ \textwidth(単カラムの場合)。どちらでも OK
%   \makebox[\textwidth][c]{%
%     \includegraphics[width=0.28\textwidth]{image/QR_100874.eps}}
%   \caption{https://vrc-ta-hub.com/}
% \end{figure}



\newpage


% ↓ このページだけ段落頭を 0pt
\begingroup

  \setlength{\parindent}{0pt}

  % 奥付を「章見出し(番号なし)」として定義し、目つぎに追加
  \phantomsection
  \addcontentsline{toc}{section}{著者紹介・奥付}
  \markright{著者紹介・奥付}


% 〜著者紹介ページの本文〜
著者紹介

% ── 夜鍋ヨナ ────────────────────────────────
{\bfseries 著者名(x@for\_your\_sns\_account)}\par      % ← \section を使わず普通の太字文字列
著者名の紹介文を書く




\vspace{1em}
\noindent\rule{\linewidth}{0.4pt}  % ← もう 1 本区切り線

\TabPositions{2em, 4em, 5em, 6em} % タブ位置を指定(任意)

タイトル \tabto{5em} :本のタイトル \tabto{20em} 本のサブタイトル \\
\tabto{20em} 第N版 第M刷

発行日 \tabto{5 em} :XXXX年MM月DD日 第N版 第M刷 発行

著者 \tabto{5em} :\columnauthor{著者名(x@for\_your\_sns\_account)}
\tabto{5em} :\columnauthor{著者名(x@for\_your\_sns\_account)}

編集 \tabto{5em} :夜鍋ヨナ(twitter@yonabeyona)

連絡先 \tabto{5em} :yonabeyona4747@gmail.com


印刷所 \tabto{5em} :株式会社○○印刷

\noindent\rule{\linewidth}{0.4pt}  % ← もう 1 本区切り線

\copyright YYYY 著者名 \copyright 2025 CS集会 \\
本書は、著者の許可なく、転載・複写・販売・公衆送信(電子媒体含む)、その他一切の無断転用を禁じます。
\endgroup            % ← ここで元の \parindent に戻る


%----裏表紙--------------------
\clearpage              % 直前のページを終わらせる
% \printnote{(裏表紙)}
\thispagestyle{empty}   % ヘッダ・フッタ(ページ番号)を消す
\null                  % なにもない本文(高さ 0 の箱)


\end{document}
